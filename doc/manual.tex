% Orpie documentation
\documentclass[11pt,notitlepage]{article}
\usepackage{times}
\usepackage{fullpage}

% End preamble
%%%%%%%%%%%%%%%%%%%%%%%%%%%%%%%%%%%%%%%%%%%%%%%%%%%%%%%%%%%%%%%%%%%%%%%%%%%%%%%
\begin{document}
\title{Orpie User Manual}
\author{Paul J. Pelzl}
\date{March 4, 2004}
\maketitle

\begin{center}
{\em ``Because the equals key is for the weak.''}
\end{center}


\section*{Introduction}
Orpie is a console-based RPN desktop calculator.  The interface is similar to
that of modern Hewlett-Packard${}^{TM}$ calculators, but has been optimized for
efficiency on a PC keyboard.  The design is also influenced to some degree by 
the Mutt email client\footnote{http://www.mutt.org} and the Vim
editor\footnote{http://vim.sf.net}.

Orpie does not have graphing capability, nor does it offer much in the way of a
programming interface; other applications such as GNU
Octave\footnote{http://www.octave.org} can already do this well.  Orpie focuses
specifically on helping you to crunch numbers quickly.

Orpie is written in Objective Caml (Ocaml)\footnote{http://caml.inria.fr/}, a
high-performance functional programming language.

\section*{Installation}
This section describes how to install Orpie by compiling from source.

Before installing Orpie, you should have installed the GNU Scientific Library
(GSL)\footnote{http://sources.redhat.com/gsl/} version 1.4 or greater.  You will
also need a curses library (e.g.
ncurses\footnote{http://www.gnu.org/software/ncurses/ncurses.html}), which is
almost certainly already installed on your system.  Finally, Ocaml 3.06 or
higher is required to compile the sources.  You will need the Nums library that
is distributed with Ocaml; at least on Debian, Nums is available as separate
packages {\tt libnums-ocaml} and {\tt libnums-ocaml-dev}.

I will assume you have received this program in the form of a source tarball, 
e.g. ``{\tt orpie-x.x.tar.gz}''.  You have undoubtedly extracted this archive 
already (e.g. using ``{\tt tar xvzf orpie-x.x.tar.gz}'').  Enter the root of 
the Orpie installation directory, e.g. ``{\tt cd orpie-x.x}''.  You can compile
the sources with the following sequence:
\begin{verbatim}
$ ./configure
$ make
\end{verbatim}
Finally, run ``{\tt make install}'' (as root) to install the executables.
``{\tt configure}'' accepts a number of parameters that you can learn about with
``{\tt ./configure --help}''.  Perhaps the most common of these is the {\tt
--prefix} option, which lets you install to a non-standard
directory\footnote{The default installation directory is {\tt /usr/local}.}.  


\section*{Quick Start}

\section*{Advanced Configuration}


\section*{Licensing}
Orpie has been made available under the GNU General Public License (GPL), 
version 2.  You should have received a copy of the GPL along with this 
program, in the file ``COPYING''.


\section*{Credits}
Orpie includes portions of the
ocamlgsl\footnote{http://oandrieu.nerim.net/ocaml/gsl/} bindings supplied by
Olivier Andrieu, as well as the curses bindings from the Ocaml Text Mode
Kit\footnote{http://www.nongnu.org/ocaml-tmk/} written by Nicolas George.  I
would like to thank these authors for helping to make Orpie possible.


\section*{Contact info}
Orpie author: Paul Pelzl {\tt <pelzlpj@eecs.umich.edu>} \\
Orpie website: {\tt http://www.eecs.umich.edu/\~{}pelzlpj/orpie} \\


\noindent
Feel free to contact me if you have bugs, feature requests, patches, etc.  I 
would also welcome volunteers interested in packaging Orpie for various platforms.

Orpie is developed with the aid of the excellent GNU Arch
RCS\footnote{http://www.gnu.org/software/gnu-arch/}.  Interested 
developers are advised to track Orpie development via my public repository: \\
\hspace*{2cm}{\tt pelzlpj@eecs.umich.edu--2004 $\backslash$ \\
\hspace*{4cm} http://www-personal.engin.umich.edu/\~{}pelzlpj/tla/2004} .  

Do you feel compelled to compensate me for writing Orpie?  As a {\em poor, 
starving} graduate student, I will gratefully accept donations.  Please see \\
{\tt http://www.eecs.umich.edu/\~{}pelzlpj/orpie/donate.html} for more information.
\end{document}


% arch-tag: DO_NOT_CHANGE_db7ed8b2-8ea4-4e32-b0f6-50482487cb00 
